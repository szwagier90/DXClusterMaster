\documentclass[]{mgr}

\usepackage{polski}

\usepackage[utf8]{inputenc}
\usepackage[T1]{fontenc}

\usepackage{graphicx}
\usepackage{caption}
\usepackage{subcaption}

\usepackage{wrapfig}
\usepackage{psfrag}

\usepackage{amsmath}
\usepackage{amsfonts}

\usepackage{listings}
\usepackage{url}

\title{System ekspertowy dopasowujący wskazania systemu DXCluster do potrzeb użytkowników}
\engtitle{Expert system for matching DXCluster system to identify the needs of users}
\author{Paweł Marcin Szwagierek}
\supervisor{dr inż. Jerzy Greblicki, I-6}
\date{2015}

\field{Informatyka (INF)}
\specialisation{Inżynieria internetowa (INT)}

\makeatletter
\def\@makechapterhead#1{%
  \vspace*{50\p@}%
  {\parindent \z@ \raggedright \normalfont
    \ifnum \c@secnumdepth >\m@ne
      \if@mainmatter
        \Huge\bfseries \thechapter.\space%
      \fi
    \fi
    \interlinepenalty\@M
    \Huge \bfseries #1\par\nobreak
    \vskip 40\p@
  }}
\makeatother

\begin{document}
    \bibliographystyle{plabbrv}
    \maketitle

    \tableofcontents

    \chapter{Problematyka amatorskich systemów radiowych}
    \label{sec:teoretical_description}
    W pierwszym rozdziale zostaną wprowadzone podstawowe pojęcia z dziedziny krótkofalarstwa, jego możliwości, zakres działań radioamatorów.

    \section{Radioamator}
    Cała dziedzina krótkofalarstwa nie miałaby sensu gdyby nie amatorzy zafascynowani radiokomunikacją, pasjonaci łączności lokalnych oraz łączności dalekiego zasięgu a także naukowcy i specjaliści ciągle udoskonalający istniejące projekty. W ogromnej większości radioamatorzy to osoby zajmujące się radiotechniką niezawodowo. Istnieją radioamatorzy bierni, poprzestający na studiowaniu samego tematu krótkofalarstwa, rzadko wkraczający w~dziedzinę praktyki radioamatorskiej. Amatorzy czynni na podstawie zezwolenia odpowiednich władz budują i~uruchamiają własne, indywidualnie lub zbiorowo (działając w~różnego rodzaju klubach), krótkofalowe i~ultrakrótkofalowe stacje nadawcze małej mocy.

    Duża część pracy radioamatora to próby, eksperymenty, budowanie swoich projektów a następnie wymiana doświadczeń z innymi. Zwykle są to wnioski na podstawie krótkich obserwacji, lecz niejednokrotnie są to duże i konkretne artykuły prezentowane w czasopismach branżowych. Wymieniane są spostrzeżenia dotyczące techniki radioamatorskiej, rozprzestrzeniania się fal elektromagnetycznych w~poszczególnych zakresach itp. Często w ten sposób społeczność radioamatorów przyczynia się do rozwoju wiedzy i~techniki. 

    \nocite{*}

\end{document}
