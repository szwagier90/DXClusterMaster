\documentclass[]{mgr}

\usepackage{polski}

\usepackage[utf8]{inputenc}
\usepackage[T1]{fontenc}

\usepackage{graphicx}
\usepackage{caption}
\usepackage{subcaption}

\usepackage{wrapfig}
\usepackage{psfrag}

\usepackage{amsmath}
\usepackage{amsfonts}

\usepackage{listings}
\usepackage{url}

\title{System ekspertowy dopasowujący wskazania systemu DXCluster do potrzeb użytkowników}
\engtitle{Expert system for matching DXCluster system to identify the needs of users}
\author{Paweł Marcin Szwagierek}
\supervisor{dr inż. Jerzy Greblicki, I-6}
\date{2015}

\field{Informatyka (INF)}
\specialisation{Inżynieria internetowa (INT)}

\makeatletter
\def\@makechapterhead#1{%
  \vspace*{50\p@}%
  {\parindent \z@ \raggedright \normalfont
    \ifnum \c@secnumdepth >\m@ne
      \if@mainmatter
        \Huge\bfseries \thechapter.\space%
      \fi
    \fi
    \interlinepenalty\@M
    \Huge \bfseries #1\par\nobreak
    \vskip 40\p@
  }}
\makeatother

\begin{document}
    \bibliographystyle{plabbrv}
    \maketitle

    \tableofcontents

    \chapter{Problematyka amatorskich systemów radiowych}
    \label{sec:teoretical_description}
    W pierwszym rozdziale zostaną wprowadzone podstawowe pojęcia z dziedziny krótkofalarstwa, jego możliwości, zakres działań radioamatorów.

    \section{Radioamator}
    Cała dziedzina krótkofalarstwa nie miałaby sensu gdyby nie amatorzy zafascynowani radiokomunikacją, pasjonaci łączności lokalnych oraz łączności dalekiego zasięgu a także naukowcy i specjaliści ciągle udoskonalający istniejące projekty. W ogromnej większości radioamatorzy to osoby zajmujące się radiotechniką niezawodowo. Istnieją radioamatorzy bierni, poprzestający na studiowaniu samego tematu krótkofalarstwa, rzadko wkraczający w~dziedzinę praktyki radioamatorskiej. Amatorzy czynni na podstawie zezwolenia odpowiednich władz budują i~uruchamiają własne, indywidualnie lub zbiorowo (działając w~różnego rodzaju klubach), krótkofalowe i~ultrakrótkofalowe stacje nadawcze małej mocy.

    Duża część pracy radioamatora to próby, eksperymenty, budowanie swoich projektów a następnie wymiana doświadczeń z innymi. Zwykle są to wnioski na podstawie krótkich obserwacji, lecz niejednokrotnie są to duże i konkretne artykuły prezentowane w czasopismach branżowych. Wymieniane są spostrzeżenia dotyczące techniki radioamatorskiej, rozprzestrzeniania się fal elektromagnetycznych w~poszczególnych zakresach itp. Często w ten sposób społeczność radioamatorów przyczynia się do rozwoju wiedzy i~techniki. 

    \section{Krótkofalarstwo}
    Terminem krótkofalarstwo określa się hobby polegające na nawiązywaniu łączności z~innymi stacjami radioamatorskimi za pomocą nadajników radiowych. W~przypadku nowoczesnych sposobów komunikacji (np. poczta elektroniczna, komunikatory internetowe, telefonia komórkowa), nieważny jest sposób przesłania informacji z~punktu A do punku~B, a sama informacja. W wypadku krótkofalarstwa większy nacisk kładzie się na sposób przesłania informacji (sprzęt radiowy, rodzaj emisji, pasmo, anteny itd) a także w wielu przypadkach na sam fakt przeprowadzonej łączności dalekiego dasięgu. Łączność ze stacją, która jest aktywna przykładowo jednego losowego dnia w roku może być bardzo dużym wyzwaniem. Podczas samych łączności krótkofalarskich istotnymi informacjami wymienianymi przez użytkowników stacji są ich osobiste znaki wywoławcze oraz raporty o~słyszalności i~sile odebranego sygnału, a~także wykorzystanych antenach, radioodbiornikach, programach komputerowych użytych do wykorzystania modulacji cyfrowych i innych parametrach związanych z~aktualnie przeprowadzaną łącznością. Samo krótkofalarstwo jest dziedziną zainteresowań o~bogatym wachlarzu możliwości.

    \subsection{Edukacja}
    (!!! COPIED !!!) Jednym z~korzyści jakie niesie ze sobą ten temat, to ogromna ilość wiedzy z~różnych lecz poniekąd pokrewnych sobie dziedzin. Taka wiedza może zostać przekazana uczniom szkół w~czasie zajęć obejmujących zagadnienia ściśle powiązane z~krótkofalarstwem. Bogatą wiedzę można także pozyskać uczestnicząc w~spotkaniach klubów krótkofalarskich lub uczestnicząc w~zajęciach prowadzonych przez takie kluby. Przekazywane tam informacje są najbardziej usystematyzowane. Rozwiązanie takie daje także możliwości skorzystania z~klubowych radiostacji. Ostatnim ze sposobów podjęcia takiej wiedzy jest nauka samodzielna korzystając z~rozmaitej literatury tematycznej, publikacji lub artykułów sporządzonych samodzielnie przez radioamatorów z~całego świata.

    Przede wszystkim radioamator uczy się przy okazji pasji dużo z~dziedziny telekomunikacji. Poznaje rodzaje anten, które wykorzystuje się w~pracy z~różnymi pasmami a także ich budowę. Uczy się także typów modulacji oraz rodzajów emisji. W~przypadku elektroniki krótkofalarstwo daje szansę poznania zasad działania różnych urządzeń elektronicznych oraz zagadnienia przetwarzania sygnałów. Budowanie automatycznych przełączników lub sterowników rotorów antenowych wiąże się z~poznaniem zagadnień z~dziedziny automatyki. Krótkofalarstwo wprowadza także wiele wiedzy z~zakresu informatyki, gdzie radioamator może wykorzystać do transmisji informacji różne rodzaje emisji cyfrowych lub stworzyć swój własny system SDR\footnote{SDR (ang. Software Defined Radio, radio programowalne) – system komunikacji radiowej, w~którym działanie podstawowych elementów elektronicznych (takich jak mieszacze, filtry, modulatory i~demodulatory) jest realizowane za pomocą programu komputerowego.} i~korzystać ze swojego oddalonego o~znaczną odległość radia za pośrednictwem Internetu lub dać możliwość korzystania ze swoich anten i~prowadzenia nasłuchu innym radioamatorom. Na pograniczu leżą także takie dziedziny nauki jak kryptografia, w~przypadku gdy użytkownicy dwóch stacji chcą się porozumiewać zachowując poufność przesyłanych informacji. Istotną umiejętnością jest także znajomość języków obcych. Bez tego prawie niemożliwym jest prowadzenie łączności z~radioamatorami posługującymi się innymi językami. Krótkofalarstwo wprowadza także wiele umiejętności stricte powiazanych z tą dziedziną - umiejętność komunikacji za pomocą telegrafii (alfabetem Morse'a), znajomość fonetycznej reprezentacji liter w różnych językach (\mbox{A -- Alpha/Adam}, B -- Bravo/Barbara, C -- Charlie/Celina, itd.).

    Jest to jedna z~nielicznych pasji, która pozwala na wykorzystanie w globalnych łącznościach własnoręcznie stworzonego sprzętu. Społeczność krótkofalowców popiera i~wręcz dopinguje budowanie i~wykorzystanie własnych urządzeń radiowych, anten a także pozostałego oprzyrządowania niezbędnego w~pracy lub ułatwiającego pracę radioamatora - jak na przykład przełączniki antenowe, wzmacniacze, klucze telegraficzne itp) oraz programów komputerowych - do transmisji danych różnymi typami emisji, do monitorowania pracy radioamatora lub do sterowania pracą radioodbiornika za pomocą komputera. Dodatkowym atutem jest brak konieczności posiadania jakichkolwiek homologacji lub zezwoleń na korzystanie z~własnych urządzeń, co jest ogromną zaletą w przeciwieństwie do pasm CB, gdzie urządzenie musi być homologowane i mieć określone warunki pracy lub pasm PMR (w jakich pracują na przykład proste urządzenia Walkie-Talkie), gdzie urządzenie musi być o bardzo niskiej mocy, mieć na stale przyczepioną antenę i trwale zamkniętą obudowę aby uniemożliwić ingerencję ze strony użytkowników. Najwięcej osób używa jednak urządzeń fabrycznych. Jest to droższe rozwiązanie, ale dużo łatwiejsze aby od razu zacząć komunikację z innymi.

    \nocite{*}

\end{document}
