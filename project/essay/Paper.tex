\documentclass[]{mgr}

\usepackage{polski}

\usepackage[utf8]{inputenc}
\usepackage[T1]{fontenc}

\usepackage{graphicx}
\usepackage{caption}
\usepackage{subcaption}

\usepackage{wrapfig}
\usepackage{psfrag}

\usepackage{amsmath}
\usepackage{amsfonts}

\usepackage{listings}
\usepackage{url}

\title{System ekspertowy dopasowujący wskazania systemu DXCluster do potrzeb użytkowników}
\engtitle{Expert system for matching DXCluster system to identify the needs of users}
\author{Paweł Marcin Szwagierek}
\supervisor{dr inż. Jerzy Greblicki, I-6}
\date{2015}

\field{Informatyka (INF)}
\specialisation{Inżynieria internetowa (INT)}

\makeatletter
\def\@makechapterhead#1{%
  \vspace*{50\p@}%
  {\parindent \z@ \raggedright \normalfont
    \ifnum \c@secnumdepth >\m@ne
      \if@mainmatter
        \Huge\bfseries \thechapter.\space%
      \fi
    \fi
    \interlinepenalty\@M
    \Huge \bfseries #1\par\nobreak
    \vskip 40\p@
  }}
\makeatother

\begin{document}
    \bibliographystyle{plabbrv}
    \maketitle

    \tableofcontents

    \chapter{Problematyka amatorskich systemów radiowych}
    \label{sec:teoretical_description}
    W pierwszym rozdziale zostaną wprowadzone podstawowe pojęcia z dziedziny krótkofalarstwa, jego możliwości, zakres działań radioamatorów.

        \section{Radioamator}
        Cała dziedzina krótkofalarstwa nie miałaby sensu gdyby nie amatorzy zafascynowani radiokomunikacją, pasjonaci łączności lokalnych oraz łączności dalekiego zasięgu a także naukowcy i specjaliści ciągle udoskonalający istniejące projekty. W ogromnej większości radioamatorzy to osoby zajmujące się radiotechniką niezawodowo. Istnieją radioamatorzy bierni, poprzestający na studiowaniu samego tematu krótkofalarstwa, rzadko wkraczający w~dziedzinę praktyki radioamatorskiej. Amatorzy czynni na podstawie zezwolenia odpowiednich władz budują i~uruchamiają własne, indywidualnie lub zbiorowo (działając w~różnego rodzaju klubach), krótkofalowe i~ultrakrótkofalowe stacje nadawcze małej mocy.

        Duża część pracy radioamatora to próby, eksperymenty, budowanie swoich projektów a następnie wymiana doświadczeń z innymi. Zwykle są to wnioski na podstawie krótkich obserwacji, lecz niejednokrotnie są to duże i konkretne artykuły prezentowane w czasopismach branżowych. Wymieniane są spostrzeżenia dotyczące techniki radioamatorskiej, rozprzestrzeniania się fal elektromagnetycznych w~poszczególnych zakresach itp. Często w ten sposób społeczność radioamatorów przyczynia się do rozwoju wiedzy i~techniki. 

        \section{Krótkofalarstwo}
        Terminem krótkofalarstwo określa się hobby polegające na nawiązywaniu łączności z~innymi stacjami radioamatorskimi za pomocą nadajników radiowych. W~przypadku nowoczesnych sposobów komunikacji (np. poczta elektroniczna, komunikatory internetowe, telefonia komórkowa), nieważny jest sposób przesłania informacji z~punktu A do punku~B, a sama informacja. W wypadku krótkofalarstwa większy nacisk kładzie się na sposób przesłania informacji (sprzęt radiowy, rodzaj emisji, pasmo, anteny itd) a także w wielu przypadkach na sam fakt przeprowadzonej łączności dalekiego dasięgu. Łączność ze stacją, która jest aktywna przykładowo jednego losowego dnia w roku może być bardzo dużym wyzwaniem. Podczas samych łączności krótkofalarskich istotnymi informacjami wymienianymi przez użytkowników stacji są ich osobiste znaki wywoławcze oraz raporty o~słyszalności i~sile odebranego sygnału, a~także wykorzystanych antenach, radioodbiornikach, programach komputerowych użytych do wykorzystania modulacji cyfrowych i innych parametrach związanych z~aktualnie przeprowadzaną łącznością. Samo krótkofalarstwo jest dziedziną zainteresowań o~bogatym wachlarzu możliwości.

            \subsection{Edukacja}
            (!!! COPIED !!!) Jednym z~korzyści jakie niesie ze sobą ten temat, to ogromna ilość wiedzy z~różnych lecz poniekąd pokrewnych sobie dziedzin. Taka wiedza może zostać przekazana uczniom szkół w~czasie zajęć obejmujących zagadnienia ściśle powiązane z~krótkofalarstwem. Bogatą wiedzę można także pozyskać uczestnicząc w~spotkaniach klubów krótkofalarskich lub uczestnicząc w~zajęciach prowadzonych przez takie kluby. Przekazywane tam informacje są najbardziej usystematyzowane. Rozwiązanie takie daje także możliwości skorzystania z~klubowych radiostacji. Ostatnim ze sposobów podjęcia takiej wiedzy jest nauka samodzielna korzystając z~rozmaitej literatury tematycznej, publikacji lub artykułów sporządzonych samodzielnie przez radioamatorów z~całego świata.

            Przede wszystkim radioamator uczy się przy okazji pasji dużo z~dziedziny telekomunikacji. Poznaje rodzaje anten, które wykorzystuje się w~pracy z~różnymi pasmami a także ich budowę. Uczy się także typów modulacji oraz rodzajów emisji. W~przypadku elektroniki krótkofalarstwo daje szansę poznania zasad działania różnych urządzeń elektronicznych oraz zagadnienia przetwarzania sygnałów. Budowanie automatycznych przełączników lub sterowników rotorów antenowych wiąże się z~poznaniem zagadnień z~dziedziny automatyki. Krótkofalarstwo wprowadza także wiele wiedzy z~zakresu informatyki, gdzie radioamator może wykorzystać do transmisji informacji różne rodzaje emisji cyfrowych lub stworzyć swój własny system SDR\footnote{SDR (ang. Software Defined Radio, radio programowalne) – system komunikacji radiowej, w~którym działanie podstawowych elementów elektronicznych (takich jak mieszacze, filtry, modulatory i~demodulatory) jest realizowane za pomocą programu komputerowego.} i~korzystać ze swojego oddalonego o~znaczną odległość radia za pośrednictwem Internetu lub dać możliwość korzystania ze swoich anten i~prowadzenia nasłuchu innym radioamatorom. Na pograniczu leżą także takie dziedziny nauki jak kryptografia, w~przypadku gdy użytkownicy dwóch stacji chcą się porozumiewać zachowując poufność przesyłanych informacji. Istotną umiejętnością jest także znajomość języków obcych. Bez tego prawie niemożliwym jest prowadzenie łączności z~radioamatorami posługującymi się innymi językami. Krótkofalarstwo wprowadza także wiele umiejętności stricte powiazanych z tą dziedziną - umiejętność komunikacji za pomocą telegrafii (alfabetem Morse'a), znajomość fonetycznej reprezentacji liter w różnych językach (\mbox{A -- Alpha/Adam}, B -- Bravo/Barbara, C -- Charlie/Celina, itd.).

            Jest to jedna z~nielicznych pasji, która pozwala na wykorzystanie w globalnych łącznościach własnoręcznie stworzonego sprzętu. Społeczność krótkofalowców popiera i~wręcz dopinguje budowanie i~wykorzystanie własnych urządzeń radiowych, anten a także pozostałego oprzyrządowania niezbędnego w~pracy lub ułatwiającego pracę radioamatora - jak na przykład przełączniki antenowe, wzmacniacze, klucze telegraficzne itp) oraz programów komputerowych - do transmisji danych różnymi typami emisji, do monitorowania pracy radioamatora lub do sterowania pracą radioodbiornika za pomocą komputera. Dodatkowym atutem jest brak konieczności posiadania jakichkolwiek homologacji lub zezwoleń na korzystanie z~własnych urządzeń, co jest ogromną zaletą w przeciwieństwie do pasm CB, gdzie urządzenie musi być homologowane i mieć określone warunki pracy lub pasm PMR (w jakich pracują na przykład proste urządzenia Walkie-Talkie), gdzie urządzenie musi być o bardzo niskiej mocy, mieć na stale przyczepioną antenę i trwale zamkniętą obudowę aby uniemożliwić ingerencję ze strony użytkowników. Najwięcej osób używa jednak urządzeń fabrycznych. Jest to droższe rozwiązanie, ale dużo łatwiejsze aby od razu zacząć komunikację z innymi.

            \subsection{Amatorska Służba Radiowa}
            Radioamatorzy posiadający urządzenia nadawczo-odbiorcze są w~stanie skomunikować się ze znaczną grupą osób w~swoim otoczeniu lub dowolnym zakątkiem kraju czy świata. To daje możliwość pomocy innym ludziom w~wymianie informacji w~sytuacjach nieprzewidzianych, nagłych wypadkach, katastrofach lub klęskach żywiołowych. Krótkofalowcy są często jedynymi, którym udaje się nawiązać łączność pomiędzy odciętymi komunikacyjnie regionami. Najlepszym przykładem może tu być przypadek powodzi jaka nawiedziła województwa Dolnośląskie oraz Opolskie w 1997 roku. Z powodu braku zasilania i awarii tradycyjnych metod komunikacji większość informacji pomiędzy ludnością cywilną przekazywana była dzięki pracy krótkofalowców. Informacje o zagnionych lub znalezionych osobach, o~stanie wody lub nadchodzących nowych zagrożeniach. Krótkofalowcy pomagali również podczas powodzi w 2007 i 2010 roku.

            Wielu radioamatorów angażuje się także w~szkolenie innych ludzi w~zakresie krótkofalarstwa lub dziedzin ściśle z~nim powiązanych. To nieoceniona pomoc dla początkujących, która pozwala łatwiej zacząć przygodę z~krótkofalarstwem i wdrożyć się tematykę bez konieczności czytania wielu potężnych tomów literatury lub czasopism branżowych.

            \subsection{Hobby}
            (!!! COPIED !!!) Najwięcej ludzi skłania się do krótkofalarstwa z~powodu hobby. Posiada wiele dyscyplin jak zawody, wyprawy, prowadzenie łączności dalekiego zasięgu, eksperymentowanie z~różnymi typami emisji lub po prostu pozostawanie w~kontakcie z~przyjaciółmi korzystając z~innego środka komunikacji od popularnych w~dzisiejszych czasach telefonii komórkowej czy Internetu.

                \subsubsection{Zawody}
                    \begin{wrapfigure}{r}{0.4\textwidth}
                        \vspace{-25pt}
                        \begin{center}
                            \includegraphics[scale=0.20]{gridsquare}
                        \end{center}
                        \vspace{-20pt}
                        \caption{Podział ziemi na kwadraty do określenia lokatora stacji}
                        \vspace{-10pt}
                        \label{fig:gridsquare}
                    \end{wrapfigure}
                Jedną z~dziedzin rywalizacji pomiędzy radioamatorami są zawody przeprowadzane w~ściśle określonym czasie. Polegają na nawiązaniu jak największej liczby łączności z~innymi operatorami stacji i~tym samym zdobywaniu punktów przydzielanych według zasad określonych w~regulaminie. Zaliczane są tylko bezbłędne łączności potwierdzone wymienionymi znakami wywoławczymi, raportami RST\footnote{RST (Readability, Strength, Tone) - raport oceniający Czytelność, Siłę oraz Ton aktualnie odbieranego sygnału. Składa się z trzech składowych, z których pierwsza (czytelność) oceniana jest w skali 1-5, druga (siła sygnału) w skali 1-9 i trzecia (ton - nieużywana w przypadku łączności fonicznych) w~\mbox{skali~1-9}. Najczęstszymi raportami są raporty 59 określające doskonałą czytelność odbieranych komunikatów i mocną siłę sygnału}, numerami porządkowymi łączności oraz lokatorami\footnote{Lokator - (ang. Grid Square Locator) – System Lokatorów, w którym świat jest podzielony na równe kwadraty. Każdy z kwadratów jest podzielony na kolejne itd. (rys.~\ref{fig:gridsquare}). Pozycja geograficzna jest zapisywana w formacie LLCCLL - gdzie L to litera a C to cyfra. Dla przykładu lokator gmachu głównego Politechniki Wrocławskiej to JO81MC.} swojej stacji. Te dane po zakończeniu zawodów poddawane są weryfikacji przez organizatora przy pomocy programów komputerowych. Zwykle nagrodami są dyplomy dla osoby lub drużyny wygrywającej zawody.

                \subsubsection{Łączności lokalne}
                    \begin{wrapfigure}{r}{0.4\textwidth}
                        \vspace{-20pt}
                        \begin{center}
                            \includegraphics[scale=0.5]{ukf_handheld}
                        \end{center}
                        \vspace{-20pt}
                        \caption{Prosty nadajnik ręczny do pracy w~paśmie UKF}
                        \vspace{-20pt}
                        \label{fig:ukf_handheld}
                    \end{wrapfigure}
                Do prowadzenia łączności ze znajomymi mieszkającymi w~naszym otoczeniu wystarczy tani nadajnik przenośny, jak pokazany na rysunku~\ref{fig:ukf_handheld} lub samochodowy pracujący na pasmach UKF. Stosunkowo mała moc urządzeń pozwala na objęcie swoim zasięgiem czasami nawet całego miasta. W~przypadku gdy taki zasięg staje się niewystarczający, w~orężu radioamatorów pozostają tzw. przemienniki - urządzenia montowane na znacznych wysokościach (wysokich obiektach miejskich, wzniesieniach terenu), które odbierają sygnał i~nadają go powtórnie z~dużo większą mocą. Skutkuje to wielokrotnym zwiększeniem zasięgu prowadzonych łączności nawet do kilkuset kilometrów. Przemienniki amatorskie nie nadają cały czas, aby nie zużywały energii gdy nikt z~nich nie korzysta. W~celu skorzystania z~takiego przemiennika należy go wcześniej ,,otworzyć'' sygnałem akustycznym o~określonej częstotliwości, jednym z~tonów DTMF lub sygnałów CTCSS.

                \subsubsection{Łączności dalekiego zasięgu (DX)}
                (!!! Copied !!!) Radioamatorzy, którzy korzystają z~bardziej zaawansowanych urządzeń i~anten mogą pracować z~innymi rodzajami emisji oraz na innych pasmach (fale krótkie (KF), fale średnie). Pozwala to na realizację łączności na dużo większe odległości - opierając się na samej mocy i~charakterystyce emitowanego sygnału, wykorzystując zjawiska pogodowe, zjawisko odbicia fal lub korzystając z~amatorskich satelitów komunikacyjnych działających w~paśmie UKF. Można dzięki temu nawiązywać łączności międzykontynentalne na dystansach rzędu tysięcy kilometrów.

                    \paragraph{Fale odbite od warstw jonosfery}
                    Łączności dalekiego zasięgu na falach krótkich można zrealizować korzystając z~nadajników małej mocy i~nieskomplikowanych drutowych anten wykorzystując zjawisko odbicia fal od jonosfery. Zjawisko to występuje powszechnie i~w~zależności od pory dnia i~aktywności słonecznej umożliwia łączności na odległość od kilkuset do kilkunastu tysięcy kilometrów. Łączności z~najbardziej odległymi stacjami prowadzi się poprzez wielokrotne odbicia, niekiedy pokonujące dłuższą drogę wokół kuli ziemskiej.

                    \paragraph{Fale odbite od powierzchni księżyca}
                    Jednym z~najbardziej niezwykłych rodzajów łączności są te z~wykorzystaniem odbicia fali od powierzchni księżyca (EME). Do tego typu łączności wykorzystuje się zaawansowane systemy antenowe (jak przedstawiona na rysunku~\ref{fig:eme_antenna} oraz nadajniki o~wysokiej mocy. Przy EME wymagana jest duża precyzja oraz doświadczenie w~prowadzeniu takich łączności, przez co jest to dziedzina krótkofalarstwa która odstrasza początkujących radioamatorów. 

                    Antena powinna być skierowana dokładnie w~kierunku księżyca, który pozostaje w~ruchu względem powierzchni ziemi. W~związku z~tym aby powszechnie wykorzystywanymi są skomplikowane systemy obracania anten. W~przypadku złego ustawienia anteny najczęstszymi zakłóceniami sygnału są zniekształcenia czoła fali w~wyniku nakładania się fal odbitych. Typowymi dla tego rodzaju łączności są zjawiska odbioru własnego echa po około 2 sekundach od wysłania komunikatu oraz niewyjaśnione dotąd echo z~dużym opóźnieniem. Skutkiem tego jest otrzymanie przez nadawcę swojego komunikatu ale po czasie dochodzącym nawet do kilku minut.

                        \begin{wrapfigure}{r}{0.4\textwidth}
                            \vspace{-20pt}
                            \begin{center}
                                \includegraphics[width=0.4\textwidth]{eme_antenna}
                            \end{center}
                            \vspace{-20pt}
                            \caption{Antena paraboliczna wykorzystywana do łączności EME pracująca w~paśmie fal 430MHz}
                            \vspace{-20pt}
                            \label{fig:eme_antenna}
                        \end{wrapfigure}

                    Łączności wykonywane tą techniką ze względu na małą skuteczność, małą pewność transmisji oraz trudności związane z~jej wykonaniem realizują tylko radioamatorzy.

                    \paragraph{Łączności satelitarne}
                    Krótkofalowcy korzystają także ze swoich satelitów. Takich satelitów niskoorbitalnych\footnote{Satelita niskoorbitalny - satelita krążący wokół Ziemi po orbicie kołowej na wysokości 10 tys. km - 500 tys. km} jest kilkadziesiąt i~co jakiś czas wysyłane są kolejne. Fale pasma UKF pozwalają na przeprowadzenie łączności z~astronautami z~międzynarodowej stacji kosmicznej ISS a także na łączności pomiędzy radioamatorami. Korzystanie z~satelitów wymaga znajomości ich pozycji na niebie, parametrów pracy oraz ciągłego korygowania ustawienia anteny. Także sama komunikacja wygląda nieco inaczej niż w~przypadku normalnej łączności. W~przypadku satelity wiadomości nadawane są na częstotliwości zwanej ,,uplink'' i~odbierane na częstotliwości oznaczonej jako ,,downlink''. Jest to uwarunkowane zasadą pracy satelity. Odbiera ona sygnały z~jednej częstotliwości i~retransmituje te same sygnały ze zwiększoną mocą na innej częstotliwości - podobnie jak to dzieje się w~przypadku przemienników amatorskich.
                        \begin{wrapfigure}{r}{0.4\textwidth}
                            \vspace{-25pt}
                            \begin{center}
                                \includegraphics[width=0.4\textwidth]{example_sat_construction}
                            \end{center}
                            \vspace{-20pt}
                            \caption{Konstrukcja radioamatora SQ5RJK do ręcznego ustawiania i~korygowania pozycji anteny podczas łączności satelitarnych}
                            \vspace{-50pt}
                            \label{fig:example_sat_construction}
                        \end{wrapfigure}
                    W~związku z~tym, że pozycja anteny musi być na bieżąco korygowana, stosuje się skomplikowane systemy obracania anten lub rolę tę przejmuje sam radioamator i~pozycję anteny koryguje ręcznie korzystając z~takich konstrukcji jak pokazana na rysunku~\ref{fig:example_sat_construction}.

                    \paragraph{Inne typy łączności}
                    Sporadycznie radioamatorzy korzystają do zrealizowania łączności z~takich zjawisk jak odbicia fal radiowych od zorzy polarnej, meteorytów a nawet chmur burzowych.

                \subsubsection{Wyprawy}
                (!!! Copied !!!) W~społeczności krótkofalarskiej popularnymi są także wyprawy zwane inaczej DXpedycjami. Radioamatorzy w~pojedynkę lub w~zorganizowanej grupie wraz z~członkami klubu zdobywają górę lub niezamieszkały teren aby tam utworzyć tymczasową stację i~przeprowadzić łączności z~innymi stacjami.

                \subsubsection{QSL}
                Podstawowym potwierdzeniem każdej wykonanej łączności jest wymiana znaków wywoławczych operatorów stacji oraz wymiana raportów słyszalności i~siły sygnału. To wystarczy, żeby uznać łączność za zrealizowaną. Oficjalnym i~bardzo eleganckim potwierdzeniem łączności są karty QSL\footnote{QSL - Jeden z~symboli Kodu Q używanego w~telegrafii i~krótkofalarstwie. Domyślnym jego znaczeniem tego symbolu jest potwierdzenie łączności.}. Są wizytówkami radioamatorów oraz kartami z~dokładnym opisem zrealizowanej łączności. Dla pewnej grupy krótkofalowców stanowią one swego rodzaju trofea ze swojej służby radiowej. Wielu z~nich skupia się na ustanawianiu łączności z~jak największą ilością stacji z~odrębnych krajów. Jest to ciekawe wyzwanie z~tego powodu, że wraz ze ,,zdobywaniem'' kolejnych krajów, stopień trudności znacząco rośnie, ponieważ istnieją kraje, w~których liczba radioamatorów jest bardzo mała, region kuli ziemskiej uznany w~społeczeństwie krótkofalarskim jako kraj jest niezamieszkały lub ze względów politycznych działalność krótkofalarska nie może funkcjonować.

    \nocite{*}

\end{document}
